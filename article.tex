\documentclass{article}
\usepackage[utf8]{inputenc}
\usepackage[T1]{fontenc}\usepackage{titlesec}

\title{Implementacja algorytmu regresji liniowej w języku Python}
\author{Karol Kluczniok}
\date{Styczeń 2023}

\begin{document}

\maketitle

\titlelabel{\thetitle.\quad}

\section{Wstęp teoretyczny}

Regresja liniowa pozwala przewidzieć wartość zmiennej na podstawie wartości innej zmiennej. Zakłada ona, że zależność pomiędzy zmienną objaśnianą a objaśniająca jest zależnością liniową. W regresji liniowej zakłada się, że wzrostowi jednej zmiennej towarzyszy wzrost lub spadek na drugiej zmiennej - tak jak w analizie korelacji.

\end{document}